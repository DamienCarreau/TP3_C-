% Tout ce qui est mis derrière un « % » n'est pas vu par LaTeX
% On appelle cela des « commentaires ».  Les commentaires permettent de
% commenter son document - comme ce que je suis en train de faire
% actuellement - et de cacher du code - cf. la ligne \pagestyle.

\documentclass[a4paper]{article}

% Options possibles : 10pt, 11pt, 12pt (taille de la fonte)
%                     oneside, twoside (recto simple, recto-verso)
%                     draft, final (stade de développement)

\usepackage[utf8]{inputenc}   % LaTeX, comprends les accents !
\usepackage[T1]{fontenc}      % Police contenant les caractères français
\usepackage[french]{babel}  % Placez ici une liste de langues                          
\usepackage{listings}
\usepackage[a4paper]{geometry}% Réduire les marges
% \pagestyle{headings}        % Pour mettre des entêtes avec les titres
                              % des sections en haut de page
\usepackage{geometry}
\geometry{hmargin=2.5cm, vmargin=1.5cm}                              
\usepackage{graphicx}
\usepackage{enumitem}
\frenchbsetup{StandardLists=true}

\title{TP C++ n$^{\circ}$2 : Gestion des entrées / sorties}           % Les paramètres du titre : titre, auteur, date
\author{ Damien Carreau \and Jérôme Hue }
                       % La date n'est pas requise (la date du
                              % jour de compilation est utilisée en son
			      % absence)

\sloppy                       % Ne pas faire déborder les lignes dans la marge

%Gestion des couleurs
\usepackage{xcolor}
 
\definecolor{codegreen}{rgb}{0,0.6,0}
\definecolor{codegray}{rgb}{0.5,0.5,0.5}
\definecolor{codepurple}{rgb}{0.58,0,0.82}
\definecolor{backcolour}{rgb}{0.95,0.95,0.92}

%pour le code
\lstdefinestyle{mystyle}{
    %backgroundcolor=\color{backcolour}, 
    frame=single,  
    %commentstyle=\color{codegreen},
    %keywordstyle=\color{magenta},
    %numberstyle=\tiny\color{codegray},
    %stringstyle=\color{codepurple},
    basicstyle=\ttfamily\footnotesize,
    breakatwhitespace=false,         
    breaklines=true,                 
    captionpos=b,                    
    keepspaces=true,                 
    numbers=left,                    
    numbersep=5pt,                  
    showspaces=false,                
    showstringspaces=false,
    showtabs=false,                  
    tabsize=2
}
\lstset{style=mystyle}

\begin{document}

\maketitle                    % Faire un titre utilisant les données
                              % passées à \title, \author et \date



\tableofcontents              % Table des matières

% \part{Titre}                % Commencer une partie...

\section{Nouvelles fonctionnalités}         % Commencer une section, etc.






\subsection{Description du fichier de sauvegarde/chargement}

Le fichier servant à charger des trajets est un fichier texte dans lequel chaque trajet est spécifié sur une ligne. Un trajet est spécifié par ses attributs séparés par des points virgules (Il y a donc n+1 élements pour n points virgules), comme suit pour un trajet simple : \\
\textit{ville de départ};\textit{ville d'arrivée};\textit{moyen de transport} \\
et pour un trajet composé : \\ 
\textit{ville de départ};\textit{ville d'arrivée};\textit{moyen de transport};\textit{ville de départ};\textit{ville d'arrivée};\textit{moyen de transport}... \\
Il n'est donc pas nécessaire de spécifier si un trajet est simple ou composé, le programme le détecte de lui même, ce qui implique qu'un trajet qui comporte 3 points-virgules ou plus est considéré comme trajet composé. Un trajet composé est composé de trajets simples \textsc{complets}, voir la sous-section suivante pour plus de détails.
\subsection{Cas particuliers}
Voici une liste de différents cas particuliers : 
\begin{itemize}
  \item \textit{ville de départ};\textit{ville d'arrivée};\textit{moyen de transport};\\
  Ici, on a terminé la liste par un point virgule. Le programme va considérer que ce trajet est composé et logiquement insérer un trajet composé d'un seul trajet simple ( De \textit{ville de départ} à \textit{ville d'arrivée} par \textit{moyen de transport}). Ce cas nécessite d'être vigilant, car si l'on charge uniquement les trajets simples, ce trajet n'est évidemment par chargé.
  \item \textit{ville de départ};\textit{ville d'arrivée};\textit{moyen de transport};\textit{ville de départ};\textit{ville d'arrivée};\textit{moyen de transport};\textit{ville de départ} \\
  Ici la ligne est bien terminée mais il manque deux éléments. Sera chargé dans le catalogue un trajet composé des deux premiers trajets simples.
  \item \textit{ville de départ};\textit{ville d'arrivée};\textit{moyen de transport};\textit{ville de départ};\textit{ville d'arrivée};\textit{moyen de transport};\textit{ville de départ};\textit{ville d'arrivée};\\
  Ce trajet comporte 9 élements, donc on charge un trajet composé de 3 trajets simples, avec le moyen de transport du dernier trajet simple non renseigné.
\end{itemize}  

Enfin, concernant l'intervalle, il est possible de choisir un intervalle plus grand que le nombre de trajets, ou bien un intervalle qui ne comporte aucun trajet.
 

\subsection{Spécification des nouvelles fonctionnalités}



\section{Problèmes rencontrés et axes d'amélioration}

\subsection{Problèmes rencontrés}
Pour gérer la collection ordonnées, on utilise un tableau dynamique. Son principal avantage est qu'il permet un accès rapide a un élément i du tableau. De plus, il n'est pas spécifié dans les consignes du TP d'implémenter une fonction permettant de supprimer un trajet du tableau, ce qui est l'un des principaux désavantages d'un tableau dynamique par rapport à une liste chaînée. Reste alors comme désavantage le fait d'avoir à copier l'intégralité du tableau lorsque celui-ci est rempli. En pratique, le tableau étant rempli à la main, cela importe peu puisque on n'atteindra pas souvent la capacité maximum du tableau.

\subsection{Axes d'amélioration et d'évolution}




% \paragraph{Titre}           % Toutes petites sections (le nom \paragraph
                              % n'est pas très bien choisi)

% \subparagraph{Titre}        % La dernière

% \appendix                   % Commençons les annexes

% \section{Titre}             % Annexe A

% \section{Titre}             % Annexe B

% \listoffigures              % Table des figures

% \listoftables               % Liste des tableaux

\end{document}